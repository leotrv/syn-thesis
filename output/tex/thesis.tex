\documentclass{article}

\title{My Thesis}
\author{Your Name}
\date{\today}

\begin{document}

\maketitle

```latex
\textbackslash\{\}section\{Methodology\}

This section outlines the methodology used to analyze the impact of Shor's algorithm on RSA and ECC, and to evaluate the performance of selected Post-Quantum Cryptography (PQC) algorithms. We will detail the mathematical description of Shor's algorithm, estimate resources needed to break RSA and ECC keys, describe the hardware and software environment used for performance testing, define the performance metrics to be measured, and explain the experimental design and data collection procedures. This research aims to quantify the vulnerability of existing RSA and ECC implementations to quantum attacks, and to assess the feasibility of using the selected PQC algorithms as replacements based on performance characteristics.

\textbackslash\{\}subsection\{Analysis of Shor's Algorithm\}

Shor's algorithm \textbackslash\{\}cite\{Shor\} is a quantum algorithm that provides a polynomial-time solution for integer factorization and the discrete logarithm problem, exponentially faster than the best-known classical algorithms. The algorithm can be broken down into the following steps:

\textbackslash\{\}begin\{enumerate\}
    \textbackslash\{\}item \textbackslash\{\}textbf\{Choose a random number:\} Select a random integer \$a\$ such that \$1 \textbackslash\{\}textless a \textbackslash\{\}textless N\$ and \$\textbackslash\{\}gcd(a, N) = 1\$, where \$N\$ is the number to be factored.
    \textbackslash\{\}item \textbackslash\{\}textbf\{Find the period:\} Use the quantum period-finding subroutine to find the period \$r\$ of the function \$f(x) = a\textasciicircum{}x \textbackslash\{\}mod N\$. This is the core quantum part of the algorithm.
    \textbackslash\{\}item \textbackslash\{\}textbf\{Check the period:\} If \$r\$ is even and \$a\textasciicircum{}\{r/2\} \textbackslash\{\}not\textbackslash\{\}equiv -1 \textbackslash\{\}pmod\{N\}\$, then compute \$\textbackslash\{\}gcd(a\textasciicircum{}\{r/2\} + 1, N)\$ and \$\textbackslash\{\}gcd(a\textasciicircum{}\{r/2\} - 1, N)\$. These are likely to be non-trivial factors of \$N\$.
    \textbackslash\{\}item \textbackslash\{\}textbf\{Repeat if necessary:\} If the factors found are trivial or if \$r\$ is odd, repeat the process with a different random \$a\$.
\textbackslash\{\}end\{enumerate\}

The quantum period-finding subroutine leverages the quantum Fourier transform (QFT) to efficiently find the period \$r\$. The algorithm creates a superposition of states, applies the function \$f(x)\$, performs a QFT, and measures the resulting state to obtain an estimate of the period.

The impact of Shor's algorithm on RSA is significant. Given the modulus \$N\$ of an RSA key, Shor's algorithm can efficiently find its prime factors \$p\$ and \$q\$, allowing an attacker to compute the private key and decrypt messages encrypted with the corresponding public key. Similarly, for ECC, Shor's algorithm can solve the discrete logarithm problem, compromising the private key used for signing and encryption.

To analyze the impact of Shor's algorithm, we will estimate the quantum resources (number of qubits and gate count) required to break RSA and ECC keys of different sizes. These estimations will be based on analytical models and, where applicable, using software simulation tools to estimate gate counts for optimized quantum circuits.  These estimations will be based on the resource analysis provided by Gidney and Eker \textbackslash\{\}cite\{GidneyEker\}, considering optimized quantum circuits for implementing Shor's algorithm. Tables and graphs will be used to visualize the relationship between key size and the required quantum resources, allowing us to assess the vulnerability of current RSA and ECC-based systems.

\textbackslash\{\}subsection\{Performance Evaluation of PQC Algorithms\}

We will select the following PQC algorithms for performance evaluation, based on their status in the NIST PQC standardization process and their potential use cases:

\textbackslash\{\}begin\{itemize\}
    \textbackslash\{\}item \textbackslash\{\}textbf\{CRYSTALS-Kyber:\} A lattice-based key-encapsulation mechanism (KEM) that offers a good balance of security and performance.
    \textbackslash\{\}item \textbackslash\{\}textbf\{CRYSTALS-Dilithium:\} A lattice-based digital signature scheme that provides strong security and efficient signature generation and verification.
\textbackslash\{\}end\{itemize\}

These algorithms represent promising candidates for replacing RSA and ECC in various applications.

The performance evaluation will be conducted on a standard desktop computer with the following specifications:

\textbackslash\{\}begin\{itemize\}
    \textbackslash\{\}item CPU: Intel Core i7-8700K (or similar)
    \textbackslash\{\}item RAM: 32 GB DDR4
    \textbackslash\{\}item Operating System: Ubuntu 20.04 LTS
    \textbackslash\{\}item Programming Language: C/C++
    \textbackslash\{\}item Cryptographic Libraries: Open Quantum Safe (OQS) project and implementations from the NIST PQC competition submissions.
\textbackslash\{\}end\{itemize\}

The following performance metrics will be measured:

\textbackslash\{\}begin\{itemize\}
    \textbackslash\{\}item \textbackslash\{\}textbf\{Key Generation Time:\} The time required to generate a key pair (public and private key).
    \textbackslash\{\}item \textbackslash\{\}textbf\{Encryption Time:\} The time required to encrypt a message using the public key (for Kyber).
    \textbackslash\{\}item \textbackslash\{\}textbf\{Decryption Time:\} The time required to decrypt a ciphertext using the private key (for Kyber).
    \textbackslash\{\}item \textbackslash\{\}textbf\{Signature Generation Time:\} The time required to generate a digital signature using the private key (for Dilithium).
    \textbackslash\{\}item \textbackslash\{\}textbf\{Signature Verification Time:\} The time required to verify a digital signature using the public key (for Dilithium).
    \textbackslash\{\}item \textbackslash\{\}textbf\{Key Size:\} The size of the public and private keys in bytes.
    \textbackslash\{\}item \textbackslash\{\}textbf\{Ciphertext Size:\} The size of the ciphertext in bytes (for Kyber).
    \textbackslash\{\}item \textbackslash\{\}textbf\{Signature Size:\} The size of the digital signature in bytes (for Dilithium).
\textbackslash\{\}end\{itemize\}

The experimental design will consist of running multiple trials (e.g., 1000) for each operation (key generation, encryption, decryption, signature generation, signature verification) and measuring the execution time using high-resolution timers. Different key sizes, specifically those recommended by NIST for security levels 1, 3, and 5, will be tested for each algorithm to analyze the relationship between key size and performance.

To ensure accuracy and validity, we will use calibrated timing functions and perform statistical analysis on the collected data. Outliers will be identified and removed to minimize the impact of external factors on the measurements. The results will be presented as averages and standard deviations, providing a comprehensive evaluation of the performance characteristics of the selected PQC algorithms. Further research could explore the performance of these algorithms on embedded systems with limited resources.
```

\end{document}
